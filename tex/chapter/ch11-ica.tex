\titlespacing*{\part}{0pt}{-20pt}{30pt} % to fix table and plot together on the first page (SEE RESTORE AT BOTTOM)
\titlespacing*{\chapter}{0pt}{-10pt}{30pt}

\begin{fullwidth}    
\part{Independent Components Analysis}
\label{part:ica}
\end{fullwidth}

\marginnote{From CS229 Spring 2021, Andrew Ng, Moses Charikar \& Christopher R\'e, Stanford University.}

Our next topic is \textbf{Independent Components Analysis} (ICA). Similar to PCA,
this will find a new basis in which to represent our data. However, the goal
is very different.

As a motivating example, consider the ``cocktail party problem.'' Here, $d$
speakers are speaking simultaneously at a party, and any microphone placed
in the room records only an overlapping combination of the $d$ speakers' voices.
But lets say we have $d$ different microphones placed in the room, and because
each microphone is a different distance from each of the speakers, it records a
different combination of the speakers' voices. Using these microphone recordings,
can we separate out the original $d$ speakers' speech signals?

To formalize this problem, we imagine that there is some data $s \in \mathbb R^d$
that is generated via $d$ independent sources. What we observe is
\[
x = As,
\]
where $A$ is an unknown square matrix called the \textbf{mixing matrix}. Repeated
observations gives us a dataset ${x^{(i)}; i = 1,\ldots,n}$, and our goal is to recover
the sources $s^{(i)}$ that had generated our data $(x^{(i)} = As^{(i)})$.

In our cocktail party problem, $s^{(i)}$ is an $d$-dimensional vector, and $s^{(i)}_j$
is the sound that speaker $j$ was uttering at time $i$. Also, $x^{(i)}$ in an $d$-dimensional
vector, and $x^{(i)}_j$ is the acoustic reading recorded by microphone $j$ at time $i$.

Let $W = A^{-1}$ be the \textbf{unmixing matrix}. Our goal is to find $W$, so
that given our microphone recordings $x^{(i)}$, we can recover the sources by
computing $s^{(i)} = Wx^{(i)}$. For notational convenience, we also let $w^\top_i$
denote
the $i$-th row of $W$, so that
\[
W = \begin{bmatrix}
    \text{--- } w_1^\top \text{ ---}\\
    \vdots\\
    \text{--- } w_d^\top \text{ ---}\\
\end{bmatrix}
\]
Thus, $w_i \in \mathbb R^d$, and the $j$-th source can be recovered as $s^{(i)}_j = w^\top_j x^{(i)}$.

\vspace{1cm}
\inlinechapter{ICA ambiguities}
To what degree can $W = A^{-1}$ be recovered? If we have no prior knowledge
about the sources and the mixing matrix, it is easy to see that there are some
inherent ambiguities in $A$ that are impossible to recover, given only the $x^{(i)}$'s.

Specifically, let $P$ be any $d$-by-$d$ permutation matrix. This means that
each row and each column of $P$ has exactly one ``$1$.'' Here are some examples
of permutation matrices:
\[
P = \begin{bmatrix}
    0 & 1 & 0\\
    1 & 0 & 0\\
    0 & 0 & 1
\end{bmatrix}; \qquad
P = \begin{bmatrix}
    0 & 1\\
    1 & 0\\
\end{bmatrix}; \qquad
P = \begin{bmatrix}
    1 & 0\\
    0 & 1\\
\end{bmatrix}.
\]
If $z$ is a vector, then $Pz$ is another vector that contains a permuted version
of $z$'s coordinates. Given only the $x^{(i)}$'s, there will be no way to distinguish
between $W$ and $PW$. Specifically, the permutation of the original sources is
ambiguous, which should be no surprise. Fortunately, this does not matter
for most applications.

Further, there is no way to recover the correct scaling of the $w_i$'s. For
instance, if $A$ were replaced with $2A$, and every $s^{(i)}$ were replaced with $(0.5)s^{(i)}$,
then our observed $x^{(i)} = 2A \cdot (0.5)s^{(i)}$ would still be the same. More broadly,
if a single column of $A$ were scaled by a factor of $\alpha$, and the corresponding
source were scaled by a factor of $1/\alpha$, then there is again no way to determine
that this had happened given only the $x^{(i)}$'s. Thus, we cannot recover the
``correct'' scaling of the sources. However, for the applications that we are
concerned with---including the cocktail party problem---this ambiguity also
does not matter. Specifically, scaling a speaker's speech signal $s^{(i)}_j$
by some positive factor $\alpha$ affects only the volume of that speaker's speech. Also, sign
changes do not matter, and $s^{(i)}_j$
and $-s^{(i)}_j$ sound identical when played on a
speaker. Thus, if the $w_i$ found by an algorithm is scaled by any non-zero real
number, the corresponding recovered source $s_i = w^\top_i x$ will be scaled by the
same factor; but this usually does not matter. (These comments also apply
to ICA for the brain/MEG data that we talked about in class.)

Are these the only sources of ambiguity in ICA? It turns out that they
are, so long as the sources $s_i$ are non-Gaussian. To see what the difficulty is
with Gaussian data, consider an example in which $n = 2$, and $s \sim \mathcal N(0,I)$.
Here, $I$ is the $2\times2$ identity matrix. Note that the contours of the density of
the standard normal distribution $\mathcal N(0,I)$ are circles centered on the origin,
and the density is rotationally symmetric.

Now, suppose we observe some $x = As$, where $A$ is our mixing matrix.
Then, the distribution of $x$ will be Gaussian, $x \sim \mathcal N(0,AA^\top)$, since
\begin{gather*}
    \mathbb E_{s\sim \mathcal N(0,I)}[x] = \mathbb E[As] = A\mathbb E[s] = 0\\
    \operatorname{Cov}[x] = \mathbb E_{s\sim \mathcal N(0,I)}[xx^\top] = \mathbb E[Ass^\top A^\top] = A\mathbb E[ss^\top]A^\top = A \cdot \operatorname{Cov}[s] \cdot A^\top = AA^\top
\end{gather*}
Now, let $R$ be an arbitrary orthogonal (less formally, a rotation/reflection)
matrix, so that $RR^\top = R^\top R = I$, and let $A' = AR$. Then if the data had
been mixed according to $A'$ instead of $A$, we would have instead observed
$x' = A's$. The distribution of $x'$ is also Gaussian, $x' \sim \mathcal N(0,AA^\top)$, since
$\mathbb E_{s\sim \mathcal N(0,I)}[x' (x' )^\top] = \mathbb E[A' ss^\top(A')^\top] = \mathbb E[ARss^\top(AR)^\top] = ARR^\top A^\top = AA^\top$.
Hence, whether the mixing matrix is $A$ or $A'$, we would observe data from
a $\mathcal N(0,AA^\top)$ distribution. Thus, there is no way to tell if the sources were
mixed using $A$ and $A'$. There is an arbitrary rotational component in the
mixing matrix that cannot be determined from the data, and we cannot
recover the original sources.

Our argument above was based on the fact that the multivariate standard
normal distribution is rotationally symmetric. Despite the bleak picture that
this paints for ICA on Gaussian data, it turns out that, so long as the data is
not Gaussian, it is possible, given enough data, to recover the $d$ independent
sources.


\vspace{1cm}
\inlinechapter{Densities and linear transformations}
Before moving on to derive the ICA algorithm proper, we first digress briefly
to talk about the effect of linear transformations on densities.

Suppose a random variable $s$ is drawn according to some density $p_s(s)$.
For simplicity, assume for now that $s \in \mathbb R$ is a real number. Now, let the
random variable $x$ be defined according to $x = As$ (here, $x \in \mathbb R,A \in \mathbb R$). Let
$p_x$ be the density of $x$. What is $p_x$?

Let $W = A^{-1}$. To calculate the ``probability'' of a particular value of $x$,
it is tempting to compute $s = Wx$, then then evaluate $p_s$ at that point, and
conclude that ``$p_x (x) = p_s(Wx)$.'' However, \textit{this is incorrect}. For example,
let $s \sim \operatorname{Uniform}[0,1]$, so $p_s(s) = \mathbb{1}\{0 \le s \le 1\}$. Now, let $A = 2$, so $x = 2s$.
Clearly, $x$ is distributed uniformly in the interval $[0,2]$. Thus, its density is
given by $p_x(x) = (0.5)\mathbb{1}\{0 \le x \le 2\}$. This does not equal $p_s(Wx)$, where
$W = 0.5 = A^{-1}$. Instead, the correct formula is $p_x (x) = p_s(Wx)|W|$.

More generally, if $s$ is a vector-valued distribution with density $p_s$, and
$x = As$ for a square, invertible matrix $A$, then the density of $x$ is given by
\[
    p_x (x) = p_s(Wx) \cdot |W|,
\]
where $W = A^{-1}$.

\paragraph{Remark} If you're seen the result that $A$ maps $[0,1]^d$ to a set of volume $|A|$,
then here's another way to remember the formula for $p_x$ given above, that also
generalizes our previous $1$-dimensional example. Specifically, let $A \in \mathbb R^{d \times d}$ be
given, and let $W = A^{-1}$ as usual. Also let $C_1 = [0,1]^d$ be the $d$-dimensional
hypercube, and define $C_2 = \{As : s \in C_1\} \subseteq \mathbb R^d$ to be the image of $C_1$
under the mapping given by $A$. Then it is a standard result in linear algebra
(and, indeed, one of the ways of defining determinants) that the volume of
$C_2$ is given by $|A|$. Now, suppose $s$ is uniformly distributed in $[0,1]^d$, so its
density is $p_s(s) = \mathbb{1}\{s \in C_1\}$. Then clearly $x$ will be uniformly distributed
in $C_2$. Its density is therefore found to be $p_x(x) = \mathbb{1}\{x \in C_2\}/\operatorname{vol}(C_2)$ (since
it must integrate over $C_2$ to $1$). But using the fact that the determinant
of the inverse of a matrix is just the inverse of the determinant, we have
$1/\operatorname{vol}(C_2) = 1/|A| = |A^{-1}| = |W|$. Thus, $p_x(x) = \mathbb{1}\{x \in C_2\}|W| = \mathbb{1}\{Wx \in
C_1\}|W| = p_s(Wx)|W|$.

\vspace{1cm}
\inlinechapter{ICA algorithm}
We are now ready to derive an ICA algorithm. We describe an algorithm
by Bell and Sejnowski, and we give an interpretation of their algorithm as a %TODO: citation.
method for maximum likelihood estimation. (This is different from their
original interpretation involving a complicated idea called the infomax principal
which is no longer necessary given the modern understanding of ICA.)

We suppose that the distribution of each source $s_j$ is given by a density
$p_s$, and that the joint distribution of the sources $s$ is given by
\[
p(s) = \prod_{j=1}^d p_s(s_j).
\]
Note that by modeling the joint distribution as a product of marginals, we
capture the assumption that the sources are independent. Using our formulas
from the previous section, this implies the following density on $x = As = W^{-1} s$:
\[
    p(x) = \prod_{j=1}^d p_s(w^\top_j x) \cdot |W|
\]
All that remains is to specify a density for the individual sources $p_s$.
Recall that, given a real-valued random variable $z$, its cumulative
distribution function (cdf) $F$ is defined by $F(z_0) = P(z \le z_0) = \int_{-\infty}^{z_0} p_z(z) dz$
and the density is the derivative of the cdf: $p_z(z) = F'(z)$.

Thus, to specify a density for the $s_i$'s, all we need to do is to specify some
cdf for it. A cdf has to be a monotonic function that increases from zero
to one. Following our previous discussion, we cannot choose the Gaussian
cdf, as ICA doesn't work on Gaussian data. What we'll choose instead as
a reasonable ``default'' cdf that slowly increases from $0$ to $1$, is the sigmoid
function $g(s) = 1/(1 + e^{-s})$. Hence, $p_s(s) = g'(s)$.\footnote{
If you have prior knowledge that the sources' densities take a certain form, then it
is a good idea to substitute that in here. But in the absence of such knowledge, the
sigmoid function can be thought of as a reasonable default that seems to work well for
many problems. Also, the presentation here assumes that either the data $x^{(i)}$ has been
preprocessed to have zero mean, or that it can naturally be expected to have zero mean
(such as acoustic signals). This is necessary because our assumption that $p_s(s) = g'(s)$
implies $\mathbb E[s] = 0$ (the derivative of the logistic function is a symmetric function, and
hence gives a density corresponding to a random variable with zero mean), which implies
$\mathbb E[x] = \mathbb E[As] = 0$.}

The square matrix $W$ is the parameter in our model. Given a training
set $\{x^{(i)} ;i = 1,\ldots,n\}$, the log likelihood is given by
\[
    \ell(W) = \sum_{i=1}^n \left( \sum_{j=1}^d \log g'(w^\top_j x^{(i)}) + \log|W| \right).
\]
We would like to maximize this in terms $W$. By taking derivatives and using
the fact (from the first set of notes) that $\nabla_W |W| = |W|(W^{-1})^\top$, we easily
derive a stochastic gradient ascent learning rule. For a training example $x^{(i)}$,
the update rule is:
\[
    W := W + \alpha \left( \begin{bmatrix}
        1 - 2g(w^\top_1 x^{(i)})\\
        1 - 2g(w^\top_2 x^{(i)})\\
        \vdots\\
        1 - 2g(w^\top_d x^{(i)})
    \end{bmatrix} x^{(i)^\top} + (W^\top)^{-1} \right),
\]
where $\alpha$ is the learning rate.

After the algorithm converges, we then compute $s^{(i)} = Wx^{(i)}$ to recover
the original sources.

\paragraph{Remark} When writing down the likelihood of the data, we implicitly
assumed that the $x^{(i)}$'s were independent of each other (for different values
of $i$; note this issue is different from whether the different coordinates of
$x^{(i)}$ are independent), so that the likelihood of the training set was given
by $\prod_i p(x^{(i)}; W)$. This assumption is clearly incorrect for speech data and
other time series where the $x^{(i)}$'s are dependent, but it can be shown that
having correlated training examples will not hurt the performance of the
algorithm if we have sufficient data. However, for problems where successive
training examples are correlated, when implementing stochastic gradient
ascent, it sometimes helps accelerate convergence if we visit training examples
in a randomly permuted order (i.e., run stochastic gradient ascent on a
randomly shuffled copy of the training set.)
